\subsection{Final comment}
In \cite{Kharrazi} are described the bases of using sensor-specific informations to identify 
the source of an image, however it was not clear which of the many analized handcrafted features were the more important for this task.
After this work it is clear that the direction is to gain a better understanding on the relevant component of the camera fingerprint.\newline
The next milestone \cite{Lukas} analizes and categorizes the sources of deterministic noise, where PRNU results to be the most relevant for solving our problem.
They also describe the way to extract the PRNU from the images, the same method is used in the reference paper \cite{Mandelli}.
Lastly in \cite{Bondi} CNN are used to extract automatically the relevant features. After the great success of NN they have been used broadly for all kind of tasks,
and many advantages are still to be discovered. In this example CNN are used without considering what has been learned about PRNU in the previous works; 
the reason for this could be that the authors wanted to find a way to learn automatically what is relevant to identify the source without rely on human knowledge (approach that has been successful in other applications),
or just because it is to be considered as a first and incomplete try to use CNN in image forensics.
Anyway all these groups helped Mandelli et al. \cite{Mandelli} to describe and implement a method that relies on all the relevant aspects of their discoveries and outperforms them.