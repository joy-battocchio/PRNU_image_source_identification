Of all the related works cited in the main paper, it has been decided to focus on three:
\begin{itemize}
    \item \textbf{Kharrazi et al.} "Blind source camera identification"\cite{Kharrazi}
    \item \textbf{Lukáš et al.} "Digital camera identification from sensor pattern noise"\cite{Lukas}
    \item \textbf{Bondi et al.} "First Steps Toward Camera Model Identification with Convolutional Neural Networks"\cite{Bondi}
\end{itemize}
The choice is motivated by the influence that these three works had on the Mandelli's one.
They all represent important milestones in the solution of the source identification problem.
For this reason it has been decided to present them in temporal order of publication, because each one of them gives an additional contribution and it's logically dependent on the previous ones.

\subsection{Kharrazi et al.}
The authors of "Blind source camera identification" propose to extract handcrafted features from images in order to use them as input for a SVM classifier.
\\The chosed features are the ones that bring more evidence of the CFA configuration and the color processing carried out by the camera, which should be unique for each sensor and therefore useful for their identification.
\\In particular they are:
\begin{itemize}
    \item \textbf{Average pixel value}. Average values in RGB channels of an image should average to gray assuming that the images has enough color variations.
    \item \textbf{RGB pairs correlation}. Capture the fact that depending on the camera structure, the correlation between different color bands could varies.
    \item \textbf{Neighbor distribution Center of mass}. Calculating the number of pixel neighbors for each pixel value, where the pixel neighbors are defined as all pixels having a difference in value of 1 or -1, from the pixel value in question. This is calculated for each color band.
    \item \textbf{RGB pairs energy ratio}. It is used in the process of white point correction.
    \item \textbf{Wavelet domain statistic}. Decompose each color band of the image using separable quadratic mirror filters and then calculate the mean for each of the 3 resulting sub-bands.
\end{itemize}
In addition to these features, different cameras produce images of different quality, so we can extract the Image Quality Metrics as features to aid in distinguishing between cameras. 
These metrics are:
\begin{itemize}
    \item \textbf{Pixel difference based measures}. Mean squared error, mean absolute error, ...
    \item \textbf{Correlation based measures}. Normalized cross correlation, ...
    \item \textbf{Spectral distance based measures}. Spectral phase and magnitude errors.
\end{itemize}
As it is written above, these features are then used by a SVM to classify the right source camera.
\textcolor{red}{insert cons??}